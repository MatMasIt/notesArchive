\documentclass{beamer}
\usepackage[utf8]{inputenc}
\usepackage{chemfig}
\usepackage{caption}
\captionsetup[figure]{labelformat=empty}% redefines the caption setup of the figures environment in the beamer class.
\usepackage[italian]{babel}
\usetheme{Madrid}
\title{Alogenazione Radicalica}
\author{Mattia Mascarello}
\beamertemplatenavigationsymbolsempty
\institute[Liceo Cocito] % (optional)
{
	
	Chimica Organica\\
	Liceo Scientifico Statale ``Leonardo Cocito''\\
	prof. Marina Orazietti

}


\date{2022}
\usepackage{cmbright}
\begin{document}
	
	\frame{\titlepage}
	
	\begin{frame}
		\frametitle{Reazione}
		È una \alert{reazione di sostituzione}, che avviene per sostituzione di atomi di $H$ con con altri atomi.\\
		È una reazione tra un alogeno e un alcano e comporta la formazione di \alert{alogenoalcani}.\\
		È composta di \textbf{3} stadi:
		\begin{itemize}
			\item \color{blue}{\hyperlink{innesco}{1° stadio (innesco)}}\\
			\item\hyperlink{propagazione}{2° stadio (propagazione)}\\
			\item\hyperlink{terminazione}{3° stadio (terminazione)}
		\end{itemize}
		\begin{block}{Alogenoalcani o alogenuri alchilici}
	Composti organici in cui uno o più atomi di idrogeno presenti in un alcano vengono sostituiti da atomi di elementi alogeni (VII gruppo: fluoro, cloro, bromo e iodio).
	\end{block}
	\end{frame}

	
	\begin{frame}[label=innesco]
		\frametitle{1° Stadio (innesco)}
		La molecola di alogeno, grazie alla luce o al calore forniti, spezza il proprio legame con una scissione  omolitica.
		I prodotti di questa reazione sono due radicali liberi di alogeno, che avendo entrambi un elettrone spaiato sono altamente reattivi
		
		\[
		A_2 \longrightarrow 2A\cdot\quad\quad 	F_2 \longrightarrow 2F\cdot\tag{1.1}
		\]
		\begin{block}{Scissione omolitica}
			Rottura di un legame covalente nella quale ognuno dei due atomi coinvolti si appropria di uno dei due elettroni prima condivisi.
			$$
			A\cdot\cdot B\longrightarrow A\cdot\quad B\cdot
			$$
			
		\end{block}
		
	\end{frame}
	\begin{frame}[label=propagazione]
		\frametitle{2° Stadio (propagazione)}
	    Il radicale (1.1) spezza il legame tra la catena carboniosa e l'atomo di idrogeno.
	    
	    Da questa reazione di ottengono un \alert{acido alogenidrico} e un \alert{radicale alchilico}.
		
		\[
	     A\cdot+ RH \longrightarrow{} HA +  R\cdot\quad\quad 	F\cdot + CH_4 \longrightarrow{} HF + \cdot CH_3   \tag{2.1}
		\]
		
		La molecola più reattiva sarà il radicale alchilico, siccome ha un elettrone spaiato.
		
		Il radicale alichilico $R\cdot$ reagisce ora con una molecola di alogeno, formando un alogenuro alchilico e un altro radicale di alogeno:
		
		 	
		 \[
		 R\cdot+ A_2 \longrightarrow{} RA +  A\cdot\quad\quad 	\cdot CH_3 + F_2   \longrightarrow{}CH_3F + F\cdot   \tag{2.2}
		 \]
		 
	\end{frame}
	\begin{frame}
	\frametitle{2° Stadio}
		\begin{figure}
		\centering
		\scalebox{.5}{
		\chemname{\chemfig{F\cdot}}{Fluoro} $\qquad+\qquad$ 
		\chemname{\chemfig{H-C(-[2]H)(-[6]H)-H}}{Metano}  $\qquad\longrightarrow{}\qquad$  \chemname{\chemfig{H-F}}{Acido fluoridrico}   $\qquad+\qquad$ 	\chemname{\chemfig{{\cdot C}(-[2]H)(-[6]H)-H}}{Metile} 
		} 
		\caption{$2.1$}
	\end{figure}
		\begin{figure}
		\centering
		\scalebox{.5}{
		\chemname{\chemfig{{\cdot C}(-[2]H)(-[6]H)-H}}{Metile}  $\qquad+\qquad$  \chemname{\chemfig{F-F}}{Fluoro} $\qquad\longrightarrow{}\qquad$  \chemname{\chemfig{H-C(-[2]H)(-[6]H)-F}}{Fluorometano}   $\qquad+\qquad$  \chemname{\chemfig{F\cdot}}{Fluoro}
	} 
	\caption{$2.2$}
	\end{figure}
	\end{frame}
	\begin{frame}[label=terminazione]
		\frametitle{3° Stadio (terminazione)}
		La reazione procede finchè tutti i radicali liberi si legano con radicali dello stesso tipo.
		
		\[
		\begin{matrix}
		A\cdot + A \cdot  \longrightarrow{} A-A & F\cdot + F\cdot  \longrightarrow{} F_2\\
		R\cdot + R\cdot  \longrightarrow{}  R-R &\qquad\qquad\quad \ \quad CH_3\cdot + CH_3\cdot  \longrightarrow{} CH_3-CH_3
		\end{matrix} \tag{3}
		\]
		
	
	\end{frame}
	\begin{frame}
	\frametitle{Prodotti}
	Essendo i radicali liberi molto reattivi, risulta difficile controllare la reazione.
	Quindi, i composti ottenuti saranno diversamente alogenati, ovvero il numero di idrogeni sostituiti varierà.
	
	\[
	CH_4+F_2 \longrightarrow{}
		CH_2F+
		CH_2F_2+
		CHF_3+
		CF_4
		\tag{4}
	\]
	\end{frame}
	\begin{frame}
	\frametitle{Prodotti}
	\begin{figure}
		\centering
	\scalebox{.47}{
	\chemname{\chemfig{H-C(-[2]H)(-[6]H)-H}}{Metano}  $\qquad+\qquad$  \chemname{\chemfig{F-F}}{Fluoro} $\qquad\longrightarrow{}\qquad$  \chemname{\chemfig{H-C(-[2]H)(-[6]H)-F}}{Fluorometano}     $\qquad+\qquad$    \chemname{\chemfig{H-C(-[2]H)(-[6]F)-F}}{Difluorometano}    $\qquad+\qquad$  \chemname{\chemfig{H-C(-[2]F)(-[6]F)-F}}{Trifluorometano} $\qquad+\qquad$  \chemname{\chemfig{F-C(-[2]F)(-[6]F)-F}}{Tetrafluorometano}
	}
	\caption{$4$}
	\end{figure}
	La reattività diminuisce all'aumentare dei periodi, essendo l'energia necessaria per la reazione progressivamente maggiore.
	\end{frame}
\end{document}
